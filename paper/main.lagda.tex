\documentclass[preprint]{sigplanconf}

\def\textmu{}
%include agda.fmt
% Packages
\usepackage{amsmath}%
\usepackage{tikz}%
\usepackage{tikz-qtree}%
\usepackage{subfigure}%
\usetikzlibrary{arrows,positioning}%
\usepackage{listings}%
\usepackage{url}%
\usepackage{wasysym}%

% Unicode support
\usepackage{textgreek}
\usepackage{ucs}
\usepackage[utf8x]{inputenc}

\DeclareUnicodeCharacter{948}{\ensuremath{\delta}}
\DeclareUnicodeCharacter{955}{\ensuremath{\lambda}}

% Coloured comments
\usepackage{color}
\usepackage{ifthen}
\newboolean{showNotes}
\newboolean{marginNotes}
\setboolean{showNotes}{false}
\setboolean{marginNotes}{false}
\newcommand{\marginNote}[1]{
\ifthenelse
  {\boolean{marginNotes}}
  {\marginpar{#1}}
  {#1}}

\newcommand{\todo}[1]{
\ifthenelse
  {\boolean{showNotes}}
  {\marginNote{\textcolor{red}{\textbf{Todo:~}#1}}}
  {}}

\newcommand{\wouter}[1]{
\ifthenelse
  {\boolean{showNotes}}
  {\marginNote{\textcolor{blue}{\textbf{Wouter:~}#1}}}
  {}}

\newcommand{\pepijn}[1]{
\ifthenelse
  {\boolean{showNotes}}
  {\marginNote{\textcolor{blue}{\textbf{Pepijn:~}#1}}}
  {}}


\begin{document}

\conferenceinfo{ICFP'14} {September 1--3, 2014, G\"oteborg, Sweden}

\title{Auto for Agda}
\subtitle{Programming proof search}

\authorinfo{Pepijn Kokke \and Wouter Swierstra}
           {Universiteit Utrecht}
           {\todo{Add pepijn's email address here.} \quad w.s.swierstra@@uu.nl}

\maketitle

\begin{abstract}
  Proof automation is important. Custom tactic languages are hacky. We
  show how proof automation can be programmed in a general purpose
  dependently typed programming language using reflection. This makes
  it easier to automate, debug, and test proof automation.\todo{Write
    good abstract}
\end{abstract}

\section{Introduction}
\label{sec:intro}

Writing proof terms in type theory is hard and often tedious.
Interactive proof assistants based on type theory, such as
Agda~\cite{agda} or Coq~\cite{coq}, take very different approaches to
facilitating this process.

The Coq proof assistant has two distinct language fragments. Besides
the programming language Gallina, there is a separate tactic language
for writing and programming proof scripts. Together with several
highly customizable tactics, the tactic language Ltac can provide
powerful proof automation~\cite{chlipala}. Having to introduce a
separate tactic language, however, seems at odds with the spirit of
type theory, where a single language is used for both proof and
computation.  Having a separate language for programming proofs has
its drawbacks. Programmers need to learn another language to automate
proofs. Debugging Ltac programs can be difficult and the resulting
proof automation may be inefficient~\cite{brabaint}.


Agda does not have Coq's segregation of proof and programming
language.  Instead, programmers are encouraged to automate proofs by
writing their own solvers~\cite{ulf-tphols}. In combination with
Agda's reflection mechanism~\cite{van-der-walt}, developers can write
powerful automatic decision procedures~\cite{allais}. Unfortunately,
not all proofs are easily automated in this fashion. When this is the
case, the user is forced to interact with the integrated development
environment and construct a proof term step by step.

This paper tries to combine the best of both worlds by implementing
tactics for proof search \emph{within} Agda itself. More specifically,
this paper makes the following novel contributions:

\begin{itemize}
\item %
  After illustrating the usage of our library with several motivating
  examples (Section~\ref{sec:motivation}), we show how to implement a
  Prolog interpreter in the style of \citet{stutterheim} in Agda
  (Section~\ref{sec:prolog}). Note that, in contrast to Agda,
  resolving a Prolog query need not terminate. Using coinduction,
  however, we can write a \emph{total} interpreter for Prolog.
\item %
  We extend this interpreter to not only produce the substitution that
  solves a Prolog goal, but also produce a proof tree
  (Section~\ref{sec:proofs}). This proof tree will serve as a witness
  for a proof found in this fashion.
\item %
  We integrate this proof search algorithm with Agda's
  \emph{reflection} mechanism (Section~\ref{sec:reflection}). This
  enables us to \emph{quote} the type of a lemma we would like to
  prove, pass this term as the goal of our proof search algorithm, and
  finally, \emph{unquote} the resulting proof term, thereby proving
  the desired lemma.
\item %
  \wouter{Example? Can we use our proof search to find out why a proof
  is not being found automatically?} \pepijn{I don't see how you would
  envision this---all we can say is ``No, all combinations of up to $d$
  of your hints fail to produce anything meaningful''---which, I suppose
  is not what you're after.}\wouter{I mean: add a 'debugging' mode to the 
  auto function, giving some trace information about the search. This can
  help figure out that we are missing a certain lemma -- this is harder to
  see in Coq using auto. See the debug trace described here 
  \url{http://adam.chlipala.net/cpdt/html/LogicProg.html}}
\item
\end{itemize}

All the code described in this paper is freely available from
GitHub\todo{add link}. It is important to emphasize that all our code
is written in the `safe' fragment of Agda: it does not depend on any
postulates or foreign functions; all definitions pass Agda's
termination checker; all metavariables are solved.


\section{Motivation}
\label{sec:motivation}

Before describing the \emph{implementation} of our library, we will
provide a brief introduction to Agda's reflection mechanism and
illustrate how the proof automation described in this paper may be
used.

\subsection*{Reflection in Agda}

Agda has a \emph{reflection} mechanism\footnote{Note that Agda's
  reflection mechanism should not be confused with the proof by
  reflection methodology.} for compile time metaprogramming in the
style of Lisp~\cite{lisp-macros}, MetaML~\cite{metaml}, and Template
Haskell~\cite{template-haskell}. This reflection mechanisms make it
possible to convert a program fragment into its corresponding abstract
syntax tree and vice versa. We will introduce Agda's reflection
mechanism here with several short examples, based on the explanation
in previous work~\cite{van-der-walt}. A more complete overview can be
found in the Agda release notes~\cite{agda-relnotes-228} and Van der
Walt's thesis~\cite{vdWalt:Thesis:2012}.

The central type in the reflection mechanism is a type |Term : Set|
that defines an abstract syntax tree for Agda terms. There are several
language constructs for quoting and unquoting such terms. The simplest
example of the reflection mechanism is the quotation of an Agda term:

\begin{code}
  idTerm : Term
  idTerm = quoteTerm (\ (x : Bool) -> x)
\end{code}

When evaluation, |idTerm| produces the following value of type |Term|:
\begin{code}
  lam visible (var 0 [])
\end{code}
On the outermost level, the |lam| constructor produces a lambda
abstraction. It has a single argument that is passed explicitly (as
opposed to Agda's implicit arguments). The body of the lambda consists
of the variable identified by the De Bruijn index 0, applied to an
empty list of arguments.

More generally, the |quote| language construct allows users to access
the internal representation of an identifier, a value of a built-in
type |Name|. Users can subsequently request the type or definition of
such names.

Dual to quotation, the |unquote| mechanism allows users to splice in a
|Term|, replacing it with a its concrete syntax. For example, we could
give a convoluted definition of the |K| combinator as follows:
\begin{code}
  const : {a b : Set} -> a  -> b -> a
  const = unquote (lam visible (lam visible (var 1 [])))
\end{code}
The language construct unquote is followed by a value of type
|Term|. In this example, we manually construct a |Term| representing
the |K| combinator and splice it in the definition of |const|.

The final piece of the reflection mechanism that we will use is the
|quoteGoal| construct. The usage of |quoteGoal| is best illustrated
with an example:
%format hole = "\AgdaHole{\{\}0}"
\begin{code}
  goalInHole : ℕ
  goalInHole = quoteGoal g in hole
\end{code}
In this example, the construct |quoteGoal g| binds the type of the
current goal, |ℕ|, to the variable |g|. When completing this
definition by filling in the hole labelled |0|, we may refer to the
variable |g|. This variable is bound to the |Term| that represents the
\emph{type} expected at the position of the |quoteGoal| statement. In
this example, |g| is bound to |def ℕ []|, the |Term| representing the
type |ℕ|.

\subsection*{Using proof automation}

To illustrate the usage of our proof tactics, we begin by defining a
predicate |Even| on natural numbers as follows:

\begin{code}
  data Even : ℕ → Set where
    Base : Even 0
    Step : ∀ {n} → Even n → Even (suc (suc n))
\end{code}
%
Next we may want to prove properties of this definition:
%
\begin{code}
  sumEven : ∀ {n m} → Even n → Even m → Even (n + m)
  sumEven Base       e2  = e2
  sumEven (Step e1)  e2  = Step (sumEven e1 e2)
\end{code}
%
As shown by Van der Walt and Swierstra~\cite{van-der-walt}, it is easy
to prove the |Even| property for closed terms using proof by
reflection. The interesting terms, however, are never closed.  For
instance, if we would like to use the |sumEven| lemma in the proof
below, we need to call it explicitly.

\begin{code}
  simple : ∀ {n} → Even n → Even (n + 2)
  simple e = sumEven e (Step Base)
\end{code}

This has its drawbacks. Manually constructing explicit proof objects
in this fashion is not easy. The proof is brittle. We cannot easily
reuse it to prove similar statements such as |Even (n + 4)|. If we
need to reformulate our statement slightly, proving |Even (2 + n)|
instead, we need to rewrite our proof. Proof automation can make
propositions more robust against such changes.

In Coq, tactics such as |auto| can be customized to handle all these
examples. Using high-level tactics makes the proof more robust against
reformulations of the exact proof statement. This paper shows how to
implement such an |auto| tactic in Agda.


Instead we would like to assemble any lemmas useful in our domain in a
hint database:

\begin{code}
  hints : HintDB
  hints = hintdb
    (quote Base :: quote Step :: quote sumEven :: [])
\end{code}

Next we use Agda's reflection mechanism to obtain a representation of
the goal that we are trying to prove. The language construct
\textbf{quoteGoal} g \textbf{in} ... binds the type of the current
goal to |g|; we then call a \emph{function} |auto|, passing our hint
database and goal as arguments.  The |auto| function tries to
construct a proof term of type |Even n → Even (n + 2)| from the hint
database. If this is successful, we unquote the proof term:

This goes beyond what is currently possible using the Agsy proof
search.  In particular, we can add hints to a database; customize the
search depth; or even implement our own search strategy.

\begin{code}
  test : ∀ {n} → Even n → Even (n + 2)
  test = quoteGoal g in unquote (auto 5 hints g)
\end{code}

Of course, such invocations of the |auto| function may fail. So what
happens if no proof exists? In that case, the |auto| function
generates a dummy term, whose type explains that the search space has
been exhausted. Unquoting this term, then gives a type error
message. For example, trying to prove |Even n → Even (n + 3)| in this
style gives the following error:

\begin{verbatim}
  Err searchSpaceExhausted !=<
    Even .n -> Even (.n + 3) of type Set
\end{verbatim}



\section{Prolog in Agda}
\label{sec:prolog}

\subsection{Terms and Rules}

The heart of our proof search implementation is the structurally recursive
unification algorithm described by~\citet{mcbride}. In this approach, terms
are encoded using finite sets for variables. This allows the algorithm to
be structurally recusive on the number of variables in a term. In addition
to this we encode constants as a |Name| with a fixed arity and a vector of
arguments.\footnote{
  In our Prolog implementation the type |Name| is left abstract, to be
  supplied by the user. The implementation of |auto| identifies it with
  an arity-annotated version of the language construct |QName|.
}

\begin{code}
  data Term (n : ℕ) : Set where
    var  : Fin n → Term n
    con  : ∀ {k} → Name k → Vec (Term n) k → Term n
\end{code}

We shall refrain from further discussion of the unification algorithm---the
interested reader can refer to~\cite{mcbride} for a detailed description.
Suffices to say that an implementation leaves us with a unifiction function
of the following type:

\begin{spec}
  unify : (t₁ t₂ : Term m) → Maybe (∃ n → Subst m n)
\end{spec}

That is, given two terms $t_1$ and $t_2$, it will compute the most general
unifier in the form of a substition that takes terms with at most $m$ free
variables to terms with at most $n$.


Next we define Prolog rules as records containing a rulename and terms for its
premises and conclusion---and again the datatype is quantified over the number of
variables used by its constituents. Note that variables are shared between the
premises and conclusion. \pepijn{Is having two types called |Name| confusing?
In the Auto module I make the distinction between |AgdaName| and |PrologName|;
should I make a distinction here between, say, |Constr| and |RuleName|?}

\begin{code}
  record Rule (n : ℕ) : Set where
    field
      name        : Name
      conclusion  : Term n
      premises    : List (Term n)
\end{code}


Lastly, before we can implement some form of proof search, we are going to need
to define injection and raiseing function for terms, rules and substitions.
We will give a definition for these function on finite sets below, and leave the
implementations for the more complex constituents up to the reader.

\begin{code}
  inject : ∀ {m} n → Fin m → Fin (m + n)
  inject n  zero    = zero
  inject n (suc i)  = suc (inject n i)

  raise : ∀ m {n} → Fin n → Fin (m + n)
  raise  zero   i  = i
  raise (suc m) i  = suc (raise m i)
\end{code}

\pepijn{This explanation is pretty terrible---any advice?}
Why do we need these functions? First, let's examine what they do for finite sets:
|inject| will send a number $i$ from a finite set of size $n$, to the \emph{same}
number in a set which is $m$ larger.
On the contrary, |raise| will send a number $i$ from a finite set of size $n$ to a
number $i + m$ in a set which is $m$ larger.

If we consider the effects of these function on |Term|s, we find that |inject|
increases the number of variables that can occur, but keeps the variable names
the same; |raise|, however, not only increases the number of variables, but also
replaces all variables with variables that are \emph{guaranteed} to be fresh.



\subsection{Proof search}

\begin{code}
  data SearchSpace (m : ℕ) : Set where
    done : (∃₂ δ n → Subst (m + δ) n) → SearchSpace m
    step : (∃ Rule → ∞ (SearchSpace m)) → SearchSpace m
\end{code}

\begin{code}
  data SearchTree (A : Set) : Set where
    fail : SearchTree A
    retn : A → SearchTree A
    fork : ∞ (List (SearchTree A)) → SearchTree A
\end{code}


\section{Constructing proof trees}
\label{sec:proofs}

\section{Adding reflection}
\label{sec:reflection}



\section{Discussion}
\label{sec:discussion}

\todo{Mention Idris}

Future work: auto rewrite; setoid rewrite; proof combinators.

limitations of using recursion in 

Combining hint data bases

Debugging failed auto attempts, or other examples from
\url{http://adam.chlipala.net/cpdt/html/LogicProg.html}

We cannot `insert goals' in the term produced by a call to auto. This
could be useful if you want to allow a tactic to return an unfinished
proof. Or can we?

Work with \emph{typed} term language. This is a hard problem.

Compare with Mtac.

\bibliographystyle{plainnat}
\bibliography{main}

\end{document}

%%% Local Variables:
%%% mode: latex
%%% TeX-master: t
%%% TeX-command-default: "rake"
%%% End:
